\documentclass[12pt,t]{beamer}
\usepackage{color}
\usepackage{amsmath}
\usepackage{paralist}
\usepackage{setspace}
\usepackage{listings}
\usepackage{graphicx}
\usepackage{geometry}
\usepackage[utf8]{inputenc}
\usepackage{etex} % Avoid an error due to a lack of registers
\usepackage[ngerman]{babel} % Defines the language of macros as well
% Desired Packages
%\usepackage[pdftex]{graphicx}
\usepackage[utf8]{inputenc}
\usepackage{amsmath}
\usepackage{amssymb}
\usepackage{lastpage}
\usepackage{listings}
\usepackage{caption}
\usepackage{xy}
\usepackage{microtype}
\usepackage{lmodern, hfoldsty}
\usepackage{ellipsis}
\usepackage{tabularx}
\usepackage{beamerthemeshadow}
\usepackage{nameref}
\usepackage{hyperref}
\usepackage{subfig}
\usepackage[absolute,overlay]{textpos}
\setlength{\TPHorizModule}{\paperwidth}
\setlength{\TPHorizModule}{\paperheight}

% Further Configurations
\newcommand{\Fat}[1]{{\large \bf \textcolor{cdc_Blue}{#1}}}
\renewcommand\rmdefault{pmy}            		% Activate Myriad Font
\definecolor{Gray}{rgb}{0.5, 0.5, 0.5}  		% Define light color
\definecolor{HighlightRed}{rgb}{0.6, 0.0, 0.0}  % Define light highlighting color
\graphicspath{{images/}}
\usetheme{UniOldbg}                     		% The main thing: our theme



\title[Theoretische Ozeanographie]{Turbulenzmodelle}
\author[Florian Börgel]{Florian Börgel}
\date{\today}
\semester{Sommersemester 2016}
\institute{Universität Oldenburg}

\begin{document}

\frame{\titlepage}
\frame{\frametitle{Gliederung}\tableofcontents}

%%%%%%%%%%%% Start of content %%%%%%%%%%%% 
\section{Task 1: Wind roses}
\begin{frame}
	\frametitle{Creation of wind roses}

\end{frame}


\begin{frame}
	\frametitle{Turbulente Längenskalen}
\begin{equation}
\epsilon\thicksim \frac{U^3}{L}
\end{equation}
\begin{equation}
\eta = \frac{\nu^3 * L}{U^3}
\end{equation}
Für $1D$ gilt also:
\begin{equation}
\frac{L}{\eta} = Re^{3/4} 
\end{equation}
(de Bruyn Kops: 8000*8000*8000)
\end{frame}

\begin{frame}
	\frametitle{Schließungsproblem der Turbulenz}
			\Fat{Vereinfachungen}:
\begin{equation}
\frac{\partial}{\partial t} = \partial_t, \hspace{0.3 cm}\frac{\partial}{\partial x} = \partial_x,\hspace{0.3 cm} \frac{\partial}{\partial x_i} = \partial_i
\end{equation}
\begin{equation}
\sum_{j=x,y,z}u_j\partial_j = u_j\partial_j
\end{equation}
\end{frame}


\begin{frame}
\frametitle{Schließungsproblem der Turbulenz}
	\Fat{Reynolds-Averaged Navier-Stokes Equations (RANS)}
\begin{equation}
\partial_t u_i + \partial_j u_ju_i = -\frac{1}{\rho}\partial_i p + \nu\partial^2_j u_i
\end{equation}
gemitteltep equation:
\begin{equation}
\partial_t \overline{u_i} + \partial_j \overline{u_ju_i} = -\frac{1}{\rho}\partial_i \overline{p} + \nu\partial^2_j \overline{u_i}
\end{equation}
Reynolds Decomposition
\begin{equation}
u = \overline{u} + u'
\end{equation}
\end{frame}

\begin{frame}
Einsetzen:
\begin{align}
\partial_t \overline{\bar{u_i}} + \partial_t \overline{u_i^{'}}+ \partial_j (\overline{\bar{u_i}*\bar{u_j}} + \overline{\bar{u_i}*u_j^{'}} + \overline{u_i^{'}*\bar{u_j}} + \overline{u_i^{'}*u_j^{'}}) \\
= - \frac{1}{\rho}*\partial_i*(\overline{\bar{p}+p^{'}}) + \nu*\partial_j^2  (\overline{\bar{u_i}+u_i^{'}})
\end{align}
Es gilt $\overline{u^{'}} = 0$, daher:
\begin{align}
\partial_t \bar{u_i} + \partial_j (\overline{u_iu_j}+{\color{red}\overline{u_i^{'}u_j{'}}}) = - \frac{1}{\rho} \partial_i \bar{p}+\nu*\partial_j^2\bar{u_i}
\end{align}
$\overline{u_i^{'}u_j{'}}$ = Reynold-Stress-Term
\end{frame}
\begin{frame}
Wie kann man den Reynold-Stress-Term beschreiben?
\begin{align}
\partial_t\overline{u_i^{'}u_j{'}}+u_k\partial_k\overline{u_i^{'}u_j{'}} =\\ -\overline{u_j^{'}u_k{'}}\partial_k\bar{u_k}
-\overline{u_i^{'}u_k{'}}\partial_k\overline{u_j}-\partial_k\overline{u_i^{'}u_j{'}u_k^{'}} \\- \frac{1}{\rho}(\partial_i\overline{u_j^{'}p^{'}}+\partial_j\overline{u_i^{'}p^{'}})+\frac{1}{\rho}(\overline{p^{'}\partial_iu_j^{'}}+\overline{p^{'}\partial_ju_i^{'}})+\nu(\partial_k^{2}\overline{u_j^{'}u_i^{'}}-2\overline{\partial_ku_i^{'}\partial_ku_j^{'}})
\end{align}

\end{frame}
%%%%%%%%%%%% End of content %%%%%%%%%%%%

\section*{Ende}

	\begin{frame}
		\frametitle{Ende}
		\begin{center}
			\Large \textbf{Vielen Dank für Ihre Aufmerksamkeit!}\\
			\large \textbf{Haben Sie Fragen?}\\
		\end{center}
	\end{frame}
	
	% Black frame to separte appendix
	\setbeamercolor{background canvas}{bg=black}
	\begin{frame}[plain]
	\end{frame}
	\setbeamercolor{background canvas}{bg=white}

%%%%%%%%%%%% Start of appendix %%%%%%%%%%%% 

%%%%%%%%%%%% End of appendix %%%%%%%%%%%%

\end{document}
